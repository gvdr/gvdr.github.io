\documentclass[ignorenonframetext,]{beamer}
\usetheme{boxes}
\usecolortheme{crane}
\setbeamertemplate{caption}[numbered]
\setbeamertemplate{caption label separator}{:}
\setbeamercolor{caption name}{fg=normal text.fg}
\usepackage{amssymb,amsmath}
\usepackage{ifxetex,ifluatex}
\usepackage{fixltx2e} % provides \textsubscript
\usepackage{lmodern}
\ifxetex
  \usepackage{fontspec,xltxtra,xunicode}
  \defaultfontfeatures{Mapping=tex-text,Scale=MatchLowercase}
  \newcommand{\euro}{€}
\else
  \ifluatex
    \usepackage{fontspec}
    \defaultfontfeatures{Mapping=tex-text,Scale=MatchLowercase}
    \newcommand{\euro}{€}
  \else
    \usepackage[T1]{fontenc}
    \usepackage[utf8]{inputenc}
      \fi
\fi
% use upquote if available, for straight quotes in verbatim environments
\IfFileExists{upquote.sty}{\usepackage{upquote}}{}
% use microtype if available
\IfFileExists{microtype.sty}{\usepackage{microtype}}{}
\usepackage{longtable,booktabs}
\usepackage{caption}
% These lines are needed to make table captions work with longtable:
\makeatletter
\def\fnum@table{\tablename~\thetable}
\makeatother

% Comment these out if you don't want a slide with just the
% part/section/subsection/subsubsection title:
\AtBeginPart{
  \let\insertpartnumber\relax
  \let\partname\relax
  \frame{\partpage}
}
\AtBeginSection{
  \let\insertsectionnumber\relax
  \let\sectionname\relax
  \frame{\sectionpage}
}
\AtBeginSubsection{
  \let\insertsubsectionnumber\relax
  \let\subsectionname\relax
  \frame{\subsectionpage}
}

\setlength{\parindent}{0pt}
\setlength{\parskip}{6pt plus 2pt minus 1pt}
\setlength{\emergencystretch}{3em}  % prevent overfull lines
\setcounter{secnumdepth}{0}

\title{Stochasticity and Evolution in Food Webs}
\author{\href{http://gvdr.github.io}{Giulio Dalla Riva}\\ University of
Canterbury, NZ}
\date{Granada Seminar June 16, 2015}

\begin{document}
\frame{\titlepage}

\begin{frame}{species ARE related}

+--------+--------+ \textbar{} Test \textbar{} Test 2 \textbar{}
+========+========+ \textbar{} Test 4 \textbar{} Test 5 \textbar{}
+--------+--------+

\begin{longtable}[c]{@{}cc@{}}
\toprule\addlinespace
ONE tree of life & ONE network of interactions
\\\addlinespace
\midrule\endhead
phylogeny & Food Web
\\\addlinespace
image & image
\\\addlinespace
\bottomrule
\end{longtable}

\end{frame}

\begin{frame}{Evolution in/of Ecology}

Evolution shaped the stochastic backbones of Food Webs

image

\end{frame}

\begin{frame}{Food Webs embedded}

\begin{block}{Random Dot Product Graphs}

image

\end{block}

\begin{block}{Phylogenetic vs.~Observed traits}

vcv(X\_e\textbar{}evolutionary model) \textasciitilde{} vcv(X)

\end{block}

\end{frame}

\begin{frame}{More questions (than answers)}

\begin{itemize}[<+->]
\itemsep1pt\parskip0pt\parsep0pt
\item
  There is phylogenetic signal
\item
  It is quite weak
\item
  It saturates with dimensionality
\item
  Fine wirings may be deceiving
\item
  Evolutionary model is inadequate
\end{itemize}

\end{frame}

\begin{frame}{(Not a) Conclusion}

\begin{itemize}[<+->]
\item
  Spoiler 1: Evolutionary distinctiveness vs.~Web Centrality
\item
  Spoiler 2: An ecological informed model of species evolution maybe
  it's (almost) there.
\end{itemize}

\end{frame}

\begin{frame}{Thanks!}

Joint work with\\Daniel B. Stouffer (University of Canterbury)

Many thanks to\\Mike Steel; Carey Priebe; A. Mooers', D.B. Stouffer's \&
J. Tylianakis's labs; \ldots{}

Funds by the Allan Wilson Centre for Molecular Ecology and Evolution.

P.S. I am looking for a research team where to develop further this
picture. Contact me if interested: gvd16@uclive.ac.nz

\end{frame}

\end{document}
